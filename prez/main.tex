\documentclass{beamer}
\mode<presentation> {
\usetheme{JuanLesPins}
\definecolor{zelena}{rgb}{.23,.74,.00}
\usecolortheme[named=zelena]{structure}
}

\usepackage{hyperref} 
\usepackage{graphicx}
\usepackage{color}
\usepackage[english,serbian]{babel}
\usepackage[utf8]{inputenc}


\def\d{{\fontencoding{T1}\selectfont\dj}}
\def\D{{\fontencoding{T1}\selectfont\DJ}}

\title[Programski jezik Lua]{Programski jezik Lua}

\author{Jovana, Jana, Katarina, Ivana}
\institute[Matematički fakultet]
{
\small{Prezentacija seminarskog rada \\u okviru kursa\\Metodologija strucnog i naucnog pisanja\\ Matematički fakultet\\}
\medskip
\textit{... ivanajordanov47@gmail.com}
}
\date{}

\begin{document}

\begin{frame}
\titlepage
\end{frame}

%------------------------------------------------

\begin{frame}
\frametitle{Sadržaj}
\tableofcontents
\end{frame}

%------------------------------------------------

\section{sekcija}
\subsection{podsekcija}
\begin{frame} 
\frametitle{podsekcija}
 testetxte text text ...

\begin{figure}
%\includegraphics[width=270pt, height=126pt]{slika0.eps}
\caption{Slikaneka}
\end{figure}

\end{frame}

%------------------------------------------------

\subsection{primer}
\begin{frame}
\frametitle{primer}

\begin{itemize}
\item sudhfudh

\item nesto

\item nesto

\item nesto nesto

\end{itemize}
\begin{figure}
%\includegraphics[width=270pt, height=106pt]{slika1.eps}
\caption{blabla}
\end{figure}
\end{frame}


%------------------------------------------------


\section{Nastanak jezika}
\subsection{Mesto nastanka i autori}
\begin{frame} 
\frametitle{Mesto nastanka i autori}

\begin{itemize}
\item Nastao 1993. na  Katoličkom univerzitetu u Rio De Žaneiru 
\item Na portugalskom znači "mesec"
\item Autori jezika: Roberto Jeruzalimski, Luiz Henrike de Figereido i Valdemar Keles

\end{itemize}

\begin{figure}
\includegraphics[width=270pt, height=126pt]{fakultet.jpg}
\caption{Katolički univerzitetu u Rio De Žaneiru}
\end{figure}

\end{frame}

%------------------------------------------------

\subsection{Prethodnici i osnovni ciljevi}
\begin{frame}
\frametitle{Prethodnici i osnovni ciljevi}

\begin{itemize}
\item Prethodnici Lue: jezici DEL i SOL

\item Jednostavna sintaksa i semantika

\item Opis podataka  kao  u  SOL-u   

\item Portabilnost  

\item  Mnogi koncepti pozajmljeni iz drugih programskih jezika

\end{itemize}
%\begin{figure}
%\includegraphics[width=270pt, height=106pt]{slika1.eps}
%\caption{blabla}
%\end{figure}
\end{frame}


%------------------------------------------------

\section{Programska okruženja}
\begin{frame} 
\frametitle{Programska okruženja}

\begin{itemize}
\item \textbf{Lapis} - HTML templating, jednostavno uvođenje middleware-a, upravljanje ORM modelima 

\item \textbf{Sailor} - mogućnost pisanja klijentskog koda u Lui, prednost izvršavanje na raznim serverima

\item \textbf{Luvit} - nalik Node.js-u, koriste istu biblioteku za asinhrone I/O operacije

\item \textbf{Fengari} - implementacija Lua virtuelne mašine pisana u JavaScriptu

\end{itemize}
%\begin{figure}
%\includegraphics[width=270pt, height=126pt]{slika0.eps}
%\caption{Slikaneka}
%\end{figure}

\end{frame}


%------------------------------------------------


\section{Zaključak}

\begin{frame}
\frametitle{Zaključak}

\begin{block}{primer teksta}
\textbf{primer naglasenog teksta}
\end{block}

\end{frame}

%------------------------------------------------

\section{Literatura}

\begin{frame}
\frametitle{Literatura}

\footnotesize{
\begin{thebibliography}{99}
\bibitem[Jordanov, 2015]{p1} Jovana, Jana, Katarina, Ivana (2019)
\newblock Programski jezik Lua \small{Seminarski rad u okviru kursa Metodologija strucnog i naucnog rada}
\bibitem[Nikic, 2004]{p1} Ime Prezime (2004)
\newblock Naziv nekog od bitnijih izvora
\end{thebibliography}
}
\end{frame}

%------------------------------------------------

\begin{frame}
\Huge{\centerline{Hvala na pažnji!}}
\end{frame}

%----------------------------------------------------------------------------------------

\end{document} 
